%% LyX 2.0.8.1 created this file.  For more info, see http://www.lyx.org/.
%% Do not edit unless you really know what you are doing.
\documentclass[english]{beamer}
\usepackage{mathptmx}
\usepackage[T1]{fontenc}
\usepackage[latin9]{inputenc}
\usepackage{amsmath}
\usepackage{amssymb}
\usepackage{graphicx}

\makeatletter

%%%%%%%%%%%%%%%%%%%%%%%%%%%%%% LyX specific LaTeX commands.
%% A simple dot to overcome graphicx limitations
\newcommand{\lyxdot}{.}


%%%%%%%%%%%%%%%%%%%%%%%%%%%%%% Textclass specific LaTeX commands.
 % this default might be overridden by plain title style
 \newcommand\makebeamertitle{\frame{\maketitle}}%
 \AtBeginDocument{
   \let\origtableofcontents=\tableofcontents
   \def\tableofcontents{\@ifnextchar[{\origtableofcontents}{\gobbletableofcontents}}
   \def\gobbletableofcontents#1{\origtableofcontents}
 }
 \long\def\lyxplainframe#1{\@lyxplainframe#1\@lyxframestop}%
 \def\@lyxplainframe{\@ifnextchar<{\@@lyxplainframe}{\@@lyxplainframe<*>}}%
 \long\def\@@lyxplainframe<#1>#2\@lyxframestop#3\lyxframeend{%
   \frame<#1>[plain]{\frametitle{#2}#3}}
 \newenvironment{centercolumns}{\begin{columns}[c]}{\end{columns}}
 \def\lyxframeend{} % In case there is a superfluous frame end

%%%%%%%%%%%%%%%%%%%%%%%%%%%%%% User specified LaTeX commands.
%\usetheme{Warsaw}
% or ...
\useoutertheme{wuerzburg}

\useinnertheme[outline]{chamfered}

\usecolortheme{shark}

\setbeamercovered{transparent}
% or whatever (possibly just delete it)

\makeatother

\usepackage{babel}
\begin{document}
\textrm{}\global\long\def\defas{\vcentcolon=}


\textrm{}\global\long\def\st{\,:\,}


\textrm{}\global\long\def\dist{\ \sim\ }


\textrm{}\global\long\def\given{\mid}


\textrm{}\global\long\def\distiid{\overset{iid}{\dist}}


\textrm{}\global\long\def\distind{\overset{ind}{\dist}}


\global\long\def\Naturals{\mathbb{N}}


\global\long\def\Rationals{\mathbb{Q}}


\global\long\def\Reals{\mathbb{R}}


\global\long\def\BorelSets{\mathcal{B}}


\global\long\def\Nats{\mathbb{N}}


\global\long\def\Ints{\mathbb{Z}}


\global\long\def\NNInts{\Ints_{+}}


\global\long\def\NNExtInts{\overline{\Ints}_{+}}


\global\long\def\Cantor{\2^{\mathbb{N}}}


\global\long\def\NNReals{\Reals_{+}}


\global\long\def\as{\textrm{ a.s.}}


\global\long\def\epi{\textrm{epi}}


\global\long\def\intr{\textrm{int}}


\global\long\def\conv{\textrm{conv}}


\global\long\def\cone{\textrm{cone}}


\global\long\def\aff{\textrm{aff}}


\global\long\def\cone{\textrm{cone}}


\global\long\def\dom{\textrm{dom}}


\global\long\def\cl{\textrm{cl}}


\global\long\def\ri{\textrm{ri}}


\global\long\def\grad{\nabla}


\global\long\def\imp{\Rightarrow}


\global\long\def\downto{\!\downarrow\!}


\global\long\def\upto{\!\uparrow\!}
\global\long\def\AND{\wedge}


\global\long\def\OR{\vee}


\global\long\def\NOT{\neg}


\global\long\def\PowerSet{\mathcal{P}}


\global\long\def\Measures{\mathcal{M}}


\global\long\def\ProbMeasures{\mathcal{M}_{1}}


\global\long\def\equaldist{\overset{d}{=}}


\global\long\def\inv{^{-1}}


\global\long\def\norm#1{\lVert#1 \rVert}


\global\long\def\event#1{\left\lbrace #1 \right\rbrace }


\global\long\def\tuple#1{\langle#1 \rangle}


\global\long\def\bspace{\Omega}


\global\long\def\bsa{\mathcal{A}}


\global\long\def\borelspace{(\bspace,\bsa)}


\global\long\def\card#1{\##1}


\global\long\def\iid{i.i.d.}


\global\long\def\gprocess#1#2{(#1)_{#2}}


\global\long\def\nprocess#1#2#3{\gprocess{#1_{#3}}{#3 \in#2}}


\global\long\def\process#1#2{\nprocess{#1}{#2}n}


\global\long\def\expect#1{\mathbb{E}\left[#1\right]}


\global\long\def\var#1{\mbox{var}\left[#1\right]}


\global\long\def\equalas{\overset{\mbox{a.s.}}{=}}


\global\long\def\iid{\distiid}


\global\long\def\abs#1{\lvert#1 \rvert}


\global\long\def\norm#1{\lVert#1\rVert}


\global\long\def\inedge#1{e_{#1}^{\mbox{in}}}


\global\long\def\outedge#1{e_{#1}^{\mbox{out}}}


\global\long\def\intd{\mathrm{d}}


\global\long\def\suchthat{\mid}


\global\long\def\Pr{P}


\global\long\def\convPr{\xrightarrow{\mbox{p}}}


\global\long\def\floor#1{\lfloor#1\rfloor}


\global\long\def\asympLim#1{\ ,#1\rightarrow\infty}


\global\long\def\PP{\Pi}


\global\long\def\PPnu{\Pi_{\nu}}


\global\long\def\vertexset#1{v\left(#1\right)}


\global\long\def\edgeset#1{e\left(#1\right)}


\global\long\def\PPbelowth#1{\Pi_{\nu,\le#1}}


\global\long\def\PPaboveth#1{\Pi_{\nu,>#1}}


\global\long\def\linverse{^{-1}}


\global\long\def\threshold{T_{\nu}}


\global\long\def\popthreshold{T_{\nu,\mbox{pop}}}


\global\long\def\popgraph{P_{\nu}}


\global\long\def\upperthreshold{T_{\nu,u}}


\global\long\def\numDegNu#1{N_{\nu,#1}}


$\global\long\def\law{\mathcal{L}}
$


\title{The general class of (sparse) random graphs arising from exchangeable
point processes}


\author{Victor~Veitch \and Daniel~M.~Roy}


\institute{Department of Statistical Sciences, University of Toronto }


\date{SSC 2016}

\makebeamertitle

\lyxframeend{}\section{Overview and Results}


\lyxframeend{}\lyxplainframe{Big Picture}
\begin{block}
{Motivating Problem}

Need families of random graphs for modelling network structures

\end{block}

\begin{block}
{Example}
\begin{itemize}
\item network: friendships among $n$ users of a social network 
\item model: family of random graphs $(G_{n})_{n\in\Nats}$ 
\end{itemize}
\end{block}

\lyxframeend{}\lyxplainframe{Sparse Networks}
\begin{block}
{Real world networks are sparsely connected}

Random graph models should be \emph{sparse}: $o(n^{2})$ edges as
number of observed nodes $n$ becomes large.

\end{block}

\pause{}
\begin{block}
{Problem}

No general framework for the statistical analysis of sparsely connected
networks.

\end{block}

\lyxframeend{}\lyxplainframe{Sparse Graphs}

We have no satisfactory answers to some fundamental questions:
\begin{enumerate}
\item how should we parameterize the space of distributions on sparse graphs? 
\item what can we learn about a large graph if we observe only a small subgraph?
and what do we mean when we say ``observe''? 
\end{enumerate}

\lyxframeend{}\lyxplainframe{Results}
\begin{block}
{Results}

We derive and study a general class of random graphs suitable for
modelling network structures.

\end{block}

\pause{}
\begin{block}
{Special cases}
\begin{itemize}
\item All dense (graphon) models
\item Caron \& Fox models
\item Sparse graphs with e.g. small world and power law behaviour
\end{itemize}
\end{block}

\lyxframeend{}\lyxplainframe{Graphs, Adjacency, and Pixel Pictures}

\begin{figure}
\begin{centering}
\includegraphics[scale=0.8]{Figures/petersen}
\par\end{centering}

\raggedleft{}{\tiny{}{[}Lov12{]}}
\end{figure}



\lyxframeend{}\lyxplainframe{Dense Graphs}
\begin{block}
{Graph models}
\begin{itemize}
\item Basic object: infinite random binary matrix $(X_{ij})$
\item Random graphs: $(G_{n})_{n}$ defined by adjacency matrix as $n\times n$
upper left submatrix
\end{itemize}
\end{block}

\pause{}
\begin{block}
{Joint exchangeability of infinite random matrices}

$(X_{ij})\equaldist(X_{\sigma(i)\sigma(j)})$ for all permutations
$\sigma\in S_{\infty}$ of the positive integers

\end{block}

\pause{}
\begin{block}
{Aldous-Hoover-Kallenberg (translated)}
\begin{itemize}
\item Primitive of inference: $W:[0,1]^{2}\to[0,1]$ (a \emph{graphon})
\item Generative model for $(X_{ij})$ in terms of $W$
\end{itemize}
\end{block}

\lyxframeend{}\lyxplainframe{Graphon Generative Model}

Given a graphon $W:[0,1]^{2}\to[0,1]$, sample a random graph by:
\begin{enumerate}
\item Assign each vertex $i$ an iid $U[0,1]$ latent random variable
\item Include each edge $(i,j)$ independently with probability $W(U_{i},U_{j})$
\end{enumerate}


\begin{figure}
\begin{centering}
\includegraphics[width=0.8\paperwidth]{Figures/graphon_figure_austere}
\par\end{centering}

Edge $(4,5)$ is included with probability $W(0.3,0.9)=W(0.9,0.3)$.
\\
The graphon $W$ is shown as a heatmap on the right.

\end{figure}



\lyxframeend{}\lyxplainframe{Representation Theorem}
\begin{block}
{Recipe for constructing statistical models}
\begin{itemize}
\item Assume a probabilistic symmetry on some infinite random structure
\item Associated representation theorem picks out privileged family of distributions
\end{itemize}
\end{block}

\begin{block}
{Examples}
\begin{itemize}
\item de Finetti's representation theorem for exchangeable sequences
\item Aldous-Hoover-Kallenberg theorem for exchangeable arrays
\end{itemize}
\end{block}

\lyxframeend{}\lyxplainframe{Caron \& Fox 2014: (Sparse) Exchangeable Graphs}
\begin{centercolumns}%{}


\column{6cm}
\begin{block}
{Key insights}
\begin{itemize}
\item adjacency matrix $\to$ \\
point process on $\NNReals^{2}$
\item array joint exchangeability $\to$ \\
point process joint exchangeability
\end{itemize}
\end{block}

\column{4cm}


\begin{figure}
\centering{}\includegraphics[width=0.5\columnwidth]{Figures/graph}\\
\includegraphics[width=1\columnwidth]{Figures/adjmeas}\\
Graph edges correspond to points on $\NNReals^{2}$
\end{figure}


\end{centercolumns}%{}

\lyxframeend{}\lyxplainframe{(Sparse) Graph Representation Theorem}
\begin{block}
<1->{Setup}
\begin{itemize}
\item Random structure: point process on $\NNReals^{2}$
\item Finite graph $G_{s}$: truncate to $s\times s$ box
\item Symmetry: joint exchangeability of point process
\end{itemize}
\end{block}

\begin{alertblock}
<2->{Representation theorem{*}}

Distribution characterized by a \emph{graphex}: a symmetric, integrable
function $W:\NNReals^{2}\to[0,1]$
\end{alertblock}

\lyxframeend{}\lyxplainframe{Graphex Model}
\begin{block}
{Generative Model}

Given $W$ an infinite random graph is sampled by:
\begin{enumerate}
\item Sample a (latent) unit rate Poisson process $\PP$ on $\boldsymbol{\theta}\times\boldsymbol{\vartheta}$.
\item For each pair of points $(\theta_{i},\vartheta_{i}),\ (\theta_{j},\vartheta_{j})\in\PP$
include edge $(\theta_{i},\theta_{j})$ with probability $W(\vartheta_{i},\vartheta_{j})$.
\item Include $\theta_{i}$ as a vertex whenever $\theta_{i}$ participates
in at least one edge. 
\end{enumerate}
\end{block}


\begin{figure}
\begin{centering}
\includegraphics[width=0.8\paperwidth]{/home/victor/Documents/Exchangeability_and_Friends/kallenberg-exchangeable-graphs/kg-figures/graphex_model_nips_talk}
\par\end{centering}

\caption{Graphex Model. \textbf{Graphex $W$ is magenta heatmap}.}
\end{figure}



\lyxframeend{}\lyxplainframe{Sampling Distribution Results}

Given graphex $W$ we know:
\begin{enumerate}
\item the expected number of vertices and edges as a function of the size
$s$
\item the asymptotic degree distribution 
\item the asymptotic connectivity structure for certain families of graphexes. 
\end{enumerate}

\begin{block}
{Punchline}

These models include a wide range of interesting graphs.
\end{block}

\lyxframeend{}\section{Sampling and Estimation}


\lyxframeend{}\lyxplainframe{Estimation}
\begin{block}
{Problem}

How can we estimate a graphex?

\end{block}

\pause{}
\begin{block}
{Setup}
\begin{itemize}
\item Observation: $G_{s}$ (and $s$)
\item Want Estimator: $\hat{W}_{G_{s}}:\NNReals^{2}\to[0,1]$
\end{itemize}
\end{block}

\lyxframeend{}\lyxplainframe{Dense Graph Estimation}
\begin{block}
{Empirical Graphon}

Let $\widetilde{W}_{n}:[0,1]^{2}\to\{0,1\}$ be the step function
corresponding to the (arbitrarily permuted) adjacency matrix. $\widetilde{W}_{n}$
is a general non-parametric estimator for graphons.

\end{block}


\begin{figure}
\begin{centering}
\includegraphics[width=0.75\paperwidth]{/home/victor/Documents/Exchangeability_and_Friends/kallenberg-exchangeable-graphs/kg-figures/OR-empirical_graphon}
\par\end{centering}

\caption{Empirical Graphon (from Orbanz Roy 2015)}
\end{figure}



\lyxframeend{}\lyxplainframe{Sparse Graph Estimation}
\begin{block}
{Empirical Graphex}

Let \textrm{$\widetilde{W}^{G_{s}}$ be the empirical graphex of $G_{s}$,
and define the empirical graphex $\widehat{W}^{G_{s}}:[0,\frac{v_{s}}{s}]\to\{0,1\}$
by 
\begin{align*}
\widehat{W}^{G_{s}}(x,y) & =\widetilde{W}^{G_{s}}(\frac{v_{s}}{s}x,\frac{v_{s}}{s}y)
\end{align*}
}
\end{block}

\pause{}
\begin{theorem}%{}
Let $\widehat{G}_{r}^{(s)}$ be generated by $\widehat{W}^{G_{s}}(x,y)$
and let $\eta_{r}^{G_{s}}$ be the (random, $G_{s}$ measurable) law
of $\eta^{G_{s}}$. Then, almost surely, 
\begin{align*}
\lim_{s\to\infty}\|\eta_{r}^{G_{s}}-\law(G_{r})\|_{\mathrm{TV}} & \to0,
\end{align*}
where $\law(G_{r})$ is the law of a size $r$ graph generated by
$W$ and $\|\cdot\|_{\mathrm{TV}}$ is the total variation distance. 


\end{theorem}%{}

\lyxframeend{}\lyxplainframe{Sparse Graph Estimation}

Estimation is also possible even when the size $s$ is unknown \\
(but it's a little bit too long to state here)


\lyxframeend{}\lyxplainframe{Summary}
\begin{block}
<1->{Summary} 
\begin{itemize}
\item Representation theorem for sparse random graphs 
\item Extends dense (graphon) theory to sparse graphs
\item Formulas for sampling distribution properties in terms of graphex
\item General non-parametric estimator
\end{itemize}
\end{block}

\begin{block}
<1->{Arxiv} 

1512.03099

\end{block}

\lyxframeend{}
\end{document}
